\documentclass{article}
\usepackage{ctex,amsmath}
\title{Homework 3}
\author{许仕杰\qquad SA170110184}
\date{Apr. 30, 2018}
\begin{document}
\maketitle
\noindent
\textbf{4.2}
(1)设$f$为时钟周期的倒数,则$T_s=W_s-f,T_p=W_p\cdot f$,由Amdahl定律可知,固定负载的加速比:
\begin{equation*}
	S=\frac{w_s+w_p}{w_s+\frac{w_p}{p}+w_o}=\frac{w_s\cdot f+w_p\cdot f}{w_s\cdot f+\frac{w_p\cdot f}{p}+w_o\cdot f}=\frac{T_1}{T_n}=\frac{CN^3}{\frac{CN^3}{n}+\frac{bN^2}{\sqrt{N}}}=\frac{nCN}{CN+b\sqrt{n}}
\end{equation*}
效率$E=\frac{S}{n}=\frac{CN}{CN+b\sqrt{n}}$,由$E$可知并行机数量越多,效率越低。原因是通信开销随节点增加而减小的速率不断变慢,与节点数不是线性相关,负载一定,节点越多,效率越低。

(2)由Custafson定律可知,固定时间的加速比:
\begin{equation*}
	S=\frac{W_s+p\cdot W_p}{W_s+W_p+W_o}=\frac{W_s\cdot f+p\cdot W_p\cdot f}{w_s\cdot f+w_p\cdot f+w_o\cdot f}=\frac{nCN^3}{CN^3+bN^2/\sqrt{n}}=\frac{nCN}{CN+b\sqrt{n}}
\end{equation*}
效率$E=\frac{S}{n}=\frac{CN}{CN+b\sqrt{n}}$,由$E$可知,时间固定,节点越多,效率越高。原因是:虽然节点数与通信开销不是线性正相关,但是随着节点数增加,并行机中的通信开销还是在减少的。

\noindent\textbf{4.11}由Amdahl定律,加速比$S=\frac{p}{1+p(p-1)}=p-1$,串行分量$f=\frac{1}{(p-1)^2}$。

\noindent\textbf{4.14}会。因为并行系统上额外时间开销是不可消除的,以等效率度量标准为例,$S=\frac{T_e}{T_p}=\frac{T_e}{\frac{T_e+T_0}{p}}=\frac{p}{1+\frac{T_0}{w}},E=\frac{S}{p}=\frac{1}{1+\frac{T_0}{w}}$,当$p$增加,$T_0$也增加,要使$E$不变,需增大$w$,因此具有良好可扩放性的并行算法,任务规模也随着问题规模的增加而增加,只不过这增长关系是线性或者亚线性的。
\end{document}
